% Options for packages loaded elsewhere
\PassOptionsToPackage{unicode}{hyperref}
\PassOptionsToPackage{hyphens}{url}
\PassOptionsToPackage{dvipsnames,svgnames,x11names}{xcolor}
%
\documentclass[
  letterpaper,
  DIV=11,
  numbers=noendperiod]{scrreprt}

\usepackage{amsmath,amssymb}
\usepackage{iftex}
\ifPDFTeX
  \usepackage[T1]{fontenc}
  \usepackage[utf8]{inputenc}
  \usepackage{textcomp} % provide euro and other symbols
\else % if luatex or xetex
  \usepackage{unicode-math}
  \defaultfontfeatures{Scale=MatchLowercase}
  \defaultfontfeatures[\rmfamily]{Ligatures=TeX,Scale=1}
\fi
\usepackage{lmodern}
\ifPDFTeX\else  
    % xetex/luatex font selection
\fi
% Use upquote if available, for straight quotes in verbatim environments
\IfFileExists{upquote.sty}{\usepackage{upquote}}{}
\IfFileExists{microtype.sty}{% use microtype if available
  \usepackage[]{microtype}
  \UseMicrotypeSet[protrusion]{basicmath} % disable protrusion for tt fonts
}{}
\makeatletter
\@ifundefined{KOMAClassName}{% if non-KOMA class
  \IfFileExists{parskip.sty}{%
    \usepackage{parskip}
  }{% else
    \setlength{\parindent}{0pt}
    \setlength{\parskip}{6pt plus 2pt minus 1pt}}
}{% if KOMA class
  \KOMAoptions{parskip=half}}
\makeatother
\usepackage{xcolor}
\setlength{\emergencystretch}{3em} % prevent overfull lines
\setcounter{secnumdepth}{5}
% Make \paragraph and \subparagraph free-standing
\ifx\paragraph\undefined\else
  \let\oldparagraph\paragraph
  \renewcommand{\paragraph}[1]{\oldparagraph{#1}\mbox{}}
\fi
\ifx\subparagraph\undefined\else
  \let\oldsubparagraph\subparagraph
  \renewcommand{\subparagraph}[1]{\oldsubparagraph{#1}\mbox{}}
\fi


\providecommand{\tightlist}{%
  \setlength{\itemsep}{0pt}\setlength{\parskip}{0pt}}\usepackage{longtable,booktabs,array}
\usepackage{calc} % for calculating minipage widths
% Correct order of tables after \paragraph or \subparagraph
\usepackage{etoolbox}
\makeatletter
\patchcmd\longtable{\par}{\if@noskipsec\mbox{}\fi\par}{}{}
\makeatother
% Allow footnotes in longtable head/foot
\IfFileExists{footnotehyper.sty}{\usepackage{footnotehyper}}{\usepackage{footnote}}
\makesavenoteenv{longtable}
\usepackage{graphicx}
\makeatletter
\def\maxwidth{\ifdim\Gin@nat@width>\linewidth\linewidth\else\Gin@nat@width\fi}
\def\maxheight{\ifdim\Gin@nat@height>\textheight\textheight\else\Gin@nat@height\fi}
\makeatother
% Scale images if necessary, so that they will not overflow the page
% margins by default, and it is still possible to overwrite the defaults
% using explicit options in \includegraphics[width, height, ...]{}
\setkeys{Gin}{width=\maxwidth,height=\maxheight,keepaspectratio}
% Set default figure placement to htbp
\makeatletter
\def\fps@figure{htbp}
\makeatother
\newlength{\cslhangindent}
\setlength{\cslhangindent}{1.5em}
\newlength{\csllabelwidth}
\setlength{\csllabelwidth}{3em}
\newlength{\cslentryspacingunit} % times entry-spacing
\setlength{\cslentryspacingunit}{\parskip}
\newenvironment{CSLReferences}[2] % #1 hanging-ident, #2 entry spacing
 {% don't indent paragraphs
  \setlength{\parindent}{0pt}
  % turn on hanging indent if param 1 is 1
  \ifodd #1
  \let\oldpar\par
  \def\par{\hangindent=\cslhangindent\oldpar}
  \fi
  % set entry spacing
  \setlength{\parskip}{#2\cslentryspacingunit}
 }%
 {}
\usepackage{calc}
\newcommand{\CSLBlock}[1]{#1\hfill\break}
\newcommand{\CSLLeftMargin}[1]{\parbox[t]{\csllabelwidth}{#1}}
\newcommand{\CSLRightInline}[1]{\parbox[t]{\linewidth - \csllabelwidth}{#1}\break}
\newcommand{\CSLIndent}[1]{\hspace{\cslhangindent}#1}

\KOMAoption{captions}{tableheading}
\makeatletter
\makeatother
\makeatletter
\@ifpackageloaded{bookmark}{}{\usepackage{bookmark}}
\makeatother
\makeatletter
\@ifpackageloaded{caption}{}{\usepackage{caption}}
\AtBeginDocument{%
\ifdefined\contentsname
  \renewcommand*\contentsname{Table of contents}
\else
  \newcommand\contentsname{Table of contents}
\fi
\ifdefined\listfigurename
  \renewcommand*\listfigurename{List of Figures}
\else
  \newcommand\listfigurename{List of Figures}
\fi
\ifdefined\listtablename
  \renewcommand*\listtablename{List of Tables}
\else
  \newcommand\listtablename{List of Tables}
\fi
\ifdefined\figurename
  \renewcommand*\figurename{Figure}
\else
  \newcommand\figurename{Figure}
\fi
\ifdefined\tablename
  \renewcommand*\tablename{Table}
\else
  \newcommand\tablename{Table}
\fi
}
\@ifpackageloaded{float}{}{\usepackage{float}}
\floatstyle{ruled}
\@ifundefined{c@chapter}{\newfloat{codelisting}{h}{lop}}{\newfloat{codelisting}{h}{lop}[chapter]}
\floatname{codelisting}{Listing}
\newcommand*\listoflistings{\listof{codelisting}{List of Listings}}
\makeatother
\makeatletter
\@ifpackageloaded{caption}{}{\usepackage{caption}}
\@ifpackageloaded{subcaption}{}{\usepackage{subcaption}}
\makeatother
\makeatletter
\@ifpackageloaded{tcolorbox}{}{\usepackage[skins,breakable]{tcolorbox}}
\makeatother
\makeatletter
\@ifundefined{shadecolor}{\definecolor{shadecolor}{rgb}{.97, .97, .97}}
\makeatother
\makeatletter
\makeatother
\makeatletter
\makeatother
\ifLuaTeX
  \usepackage{selnolig}  % disable illegal ligatures
\fi
\IfFileExists{bookmark.sty}{\usepackage{bookmark}}{\usepackage{hyperref}}
\IfFileExists{xurl.sty}{\usepackage{xurl}}{} % add URL line breaks if available
\urlstyle{same} % disable monospaced font for URLs
\hypersetup{
  pdftitle={The EMF Research Book (draft edition)},
  pdfauthor={Mads Rohde},
  colorlinks=true,
  linkcolor={blue},
  filecolor={Maroon},
  citecolor={Blue},
  urlcolor={Blue},
  pdfcreator={LaTeX via pandoc}}

\title{The EMF Research Book (draft edition)}
\usepackage{etoolbox}
\makeatletter
\providecommand{\subtitle}[1]{% add subtitle to \maketitle
  \apptocmd{\@title}{\par {\large #1 \par}}{}{}
}
\makeatother
\subtitle{A collection of theories and hyptheses for the study of
biological and health effects of electromagnetic fields}
\author{Mads Rohde}
\date{2024-11-09}

\begin{document}
\maketitle
\ifdefined\Shaded\renewenvironment{Shaded}{\begin{tcolorbox}[enhanced, sharp corners, boxrule=0pt, breakable, borderline west={3pt}{0pt}{shadecolor}, interior hidden, frame hidden]}{\end{tcolorbox}}\fi

\renewcommand*\contentsname{Table of contents}
{
\hypersetup{linkcolor=}
\setcounter{tocdepth}{2}
\tableofcontents
}
\bookmarksetup{startatroot}

\hypertarget{preface}{%
\chapter*{Preface}\label{preface}}
\addcontentsline{toc}{chapter}{Preface}

\markboth{Preface}{Preface}

This is a collection of theories and hypotheses related to the study of
biological and health effects of non-ionizing radiation.

The book is an open science project started in 2024 and is work in
progress. A final edition may take years to complete. Scientists and
authors familiar with the topic can contribute and will be credited for
any contribution. If you want to contribute, you can simply submit
additions and revisions at the book's
\href{https://github.com/cortexR/emf-research-book/}{GitHub repository}.
Currently, I serve as the editor, but that may be changed if more
experienced authors decide to contribute.

The book at this page will remain free. The book is licensed under
\href{https://creativecommons.org/licenses/by-nc-nd/4.0/deed.en}{CC
BY-NC-ND 4.0}. Potentially, additional paperback, hard copy or e-book
editions can be made available for sale at a future point in time.

The aim of the book is to be a comprehensive list of the theories and
hypotheses related to non-ionizing radiation that can be used for
researchers and others who want to familiar themselves with the topic;
look up potential frameworks to interpret findings; or explore research
gaps and possible new research questions and methods.

The aim is not to discuss specific research findings, but rather to
provide a scope of theories. However, relevant scientific references for
each theory should be included. But to support the progress of the
development of this book, such lists may most often not be exhaustive,
meaning that some a theory may be included with one reference initially
that is not necessarily the most updated and correct reference.

It is also not the aim that theories or hypotheses included should have
been proven to be correct (or scientifically speaking,not haven been
falsified). Theories that are speculative, or even wrong, may also be
included, and the reader should be aware this. The reason for this is
that ideas that have not yet been researched may have extra value in
that they may point to research gaps, and theories that have proven
wrong may prohibit other researchers to waste their efforst following
that same path.

\bookmarksetup{startatroot}

\hypertarget{introduction-draft}{%
\chapter{Introduction (draft)}\label{introduction-draft}}

Electromagnetic fields are \ldots{}

\part{Characteristics (draft)}

\hypertarget{characteristics-of-electromagnetic-fields-and-radiation}{%
\section*{Characteristics of electromagnetic fields and
radiation}\label{characteristics-of-electromagnetic-fields-and-radiation}}
\addcontentsline{toc}{section}{Characteristics of electromagnetic fields
and radiation}

\markright{Characteristics of electromagnetic fields and radiation}

\hypertarget{nomenclature-draft}{%
\chapter{Nomenclature (draft)}\label{nomenclature-draft}}

\part{Effects (draft)}

\hypertarget{thermal-theories-draft}{%
\chapter{Thermal theories (draft)}\label{thermal-theories-draft}}

\hypertarget{non-thermal-theories-draft}{%
\chapter{Non-thermal theories
(draft)}\label{non-thermal-theories-draft}}

\hypertarget{the-interaction-with-specific-biological-components}{%
\section{The interaction with specific biological
components}\label{the-interaction-with-specific-biological-components}}

\hypertarget{melanin}{%
\subsection{Melanin}\label{melanin}}

\hypertarget{the-zeeman-effect-chiabrerazeemanstarkmodelingrf2000}{%
\section{The Zeeman effect (Chiabrera et al.
2000)}\label{the-zeeman-effect-chiabrerazeemanstarkmodelingrf2000}}

\hypertarget{ionizing-radiation-theories-draft}{%
\chapter{Ionizing radiation theories
(draft)}\label{ionizing-radiation-theories-draft}}

\hypertarget{psychological-theories-draft}{%
\chapter{Psychological theories
(draft)}\label{psychological-theories-draft}}

\part{Resources (draft)}

In this part of the book, various resources that can aid in research on
electromagnetic fields. The various chapters holds resources such as
information on literature databases, open data and scientific
communities around the world.

\hypertarget{idenifying-research-gaps}{%
\section*{Idenifying research gaps}\label{idenifying-research-gaps}}
\addcontentsline{toc}{section}{Idenifying research gaps}

\markright{Idenifying research gaps}

If you are interested in conducting new research on EMFs, the table
below with \emph{the seven research gaps} from Miles (2017) may aid you.
What research gap \emph{type} do you find to be the most important
within EMF science?

\begin{longtable}[]{@{}
  >{\raggedright\arraybackslash}p{(\columnwidth - 2\tabcolsep) * \real{0.2351}}
  >{\raggedright\arraybackslash}p{(\columnwidth - 2\tabcolsep) * \real{0.7649}}@{}}
\toprule\noalign{}
\begin{minipage}[b]{\linewidth}\raggedright
Research Gap Type
\end{minipage} & \begin{minipage}[b]{\linewidth}\raggedright
Definition
\end{minipage} \\
\midrule\noalign{}
\endhead
\bottomrule\noalign{}
\endlastfoot
Evidence Gap (Contradictory Evidence Gap) & Results from studies allow
for conclusions in their own right, but are contradictory when examined
from a more abstract point of view {[}Jacobs, 2011; Müller-Bloch \&
Kranz, 2014; Miles, 2017{]}. \\
Knowledge Gap (Knowledge Void Gap) & Desired research findings do not
exist {[}Jacobs, 2011; Müller-Bloch \& Kranz, 2014; Miles, 2017{]}. \\
Practical-Knowledge Gap (Action-Knowledge Conflict Gap) & Professional
behavior or practices deviate from research findings or are not covered
by research {[}Jacobs, 2011; Müller-Bloch \& Kranz, 2014; Miles,
2017{]}. \\
Methodological Gap (Method and Research Design Gap) & A variation of
research methods is necessary to generate new insights or to avoid
distorted findings {[}Jacobs, 2011; Müller-Bloch \& Kranz, 2014; Miles,
2017{]}. \\
Empirical Gap (Evaluation Void Gap) & Research findings or propositions
need to be evaluated or empirically verified {[}Jacobs, 2011;
Müller-Bloch \& Kranz, 2014; Miles, 2017{]}. \\
Theoretical Gap (Theory Application Void Gap) & Theory should be applied
to certain research issues to generate new insights. There is lack of
theory thus a gap exists {[}Müller-Bloch \& Kranz, 2014; Jacobs, 2011;
Müller-Bloch \& Kranz, 2014; Miles, 2017{]}. \\
Population Gap & Research regarding the population that is not
adequately represented or under-researched in the evidence base of prior
research (e.g., gender, race/ethnicity, age, etc.) {[}Robinson, et al,
2011{]}. \\
Sources & Robinson, Saldanhea, \& McKoy (2011); Müller-Bloch \& Kranz,
(2015); Miles, (2017). \\
\end{longtable}

\hypertarget{literature-data-draft}{%
\chapter{Literature data (draft)}\label{literature-data-draft}}

\hypertarget{soviet-research}{%
\section{Soviet research}\label{soviet-research}}

In the earliest time of EMF science, much research was done in the
Soviet Union.

Big portions of this body of research may not be available in literature
databases. As a starting point of an exploration of the Soviet research,
below is a list of some sources discussing this literature.

\begin{itemize}
\tightlist
\item
  Glaser and Dodge (1976)
\item
  D. I. McRee (1979)
\item
  Donald I. McRee (1980)
\item
  Kositsky, Nizhelska, and Ponezha (2001)
\item
  Kostoff (2019)
\item
  Kostoff (2020)
\end{itemize}

\hypertarget{open-data-draft}{%
\chapter{Open data (draft)}\label{open-data-draft}}

\hypertarget{emf-research-around-the-world-draft-edition}{%
\chapter{EMF research around the world (draft
edition)}\label{emf-research-around-the-world-draft-edition}}

\bookmarksetup{startatroot}

\hypertarget{references}{%
\chapter*{References}\label{references}}
\addcontentsline{toc}{chapter}{References}

\markboth{References}{References}

\hypertarget{refs}{}
\begin{CSLReferences}{1}{0}
\leavevmode\vadjust pre{\hypertarget{ref-chiabreraZeemanStarkModelingRF2000}{}}%
Chiabrera, A., B. Bianco, E. Moggia, and J. J. Kaufman. 2000.
{``Zeeman-{Stark} Modeling of the {RF EMF} Interaction with Ligand
Binding.''} \emph{Bioelectromagnetics} 21 (4): 312--24.
\url{https://doi.org/10.1002/(sici)1521-186x(200005)21:4\%3C312::aid-bem7\%3E3.0.co;2-\#}.

\leavevmode\vadjust pre{\hypertarget{ref-glaserBiomedicalAspectsRadiofrequency1976}{}}%
Glaser, Zorach R., and Christopher H. Dodge. 1976. {``Biomedical Aspects
of Radiofrequency and Microwave Radiation: A Review of Selected
{Soviet}, {East European}, and {Western} References.''} In
\emph{Biologic {Effects} of {Electromagnetic Waves}: {Selected Papers}
of the {USNC}/{URSI Annual Meeting} ({Boulder}, {Colorado}, {Oct}.
20-23, 1975)}, 2--34. {HEW Publications (FDA) 77-8010 and 77-8011}.

\leavevmode\vadjust pre{\hypertarget{ref-kositskyInfluenceHighfrequencyElectromagnetic2001a}{}}%
Kositsky, Nikolai Nikolaevich, Aljona Igorevna Nizhelska, and Grigory
Vasil'evich Ponezha. 2001. {``Influence of High-Frequency
Electromagnetic Radiation at Non-Thermal Intensities on the Human
Body.''} \emph{No Place To Hide-Newsletter of the Cellular Phone
Taskforce Inc} 3 (1): 1--33.

\leavevmode\vadjust pre{\hypertarget{ref-kostoffAdverseEffectsWireless2019}{}}%
Kostoff, Ronald N. 2019. {``Adverse {Effects} of {Wireless
Radiation}.''}

\leavevmode\vadjust pre{\hypertarget{ref-kostoffLargestUnethicalMedical2020}{}}%
---------. 2020. {``The {Largest Unethical Medical Experiment} in {Human
History}.''}

\leavevmode\vadjust pre{\hypertarget{ref-mcreeReviewSovietEastern1979}{}}%
McRee, D. I. 1979.
{``\href{https://www.ncbi.nlm.nih.gov/pmc/articles/PMC1807746}{Review of
{Soviet}/{Eastern European} Research on Health Aspects of Microwave
Radiation}.''} \emph{Bulletin of the New York Academy of Medicine} 55
(11): 1133--51.

\leavevmode\vadjust pre{\hypertarget{ref-mcreeSovietEasternEuropean1980}{}}%
McRee, Donald I. 1980. {``Soviet and {Eastern European} Research on
Biological Effects of Microwave Radiation.''} \emph{Proceedings of the
IEEE} 68 (1): 84--91.

\leavevmode\vadjust pre{\hypertarget{ref-milesTaxonomyResearchGaps2017}{}}%
Miles, D. Anthony. 2017. {``A Taxonomy of Research Gaps: {Identifying}
and Defining the Seven Research Gaps.''} In \emph{Doctoral Student
Workshop: Finding Research Gaps-Research Methods and Strategies,
{Dallas}, {Texas}}, 1--15.

\end{CSLReferences}



\end{document}
